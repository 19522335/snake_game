\documentclass{article}

\usepackage[utf8]{inputenc}
\usepackage[utf8]{vietnam}
\usepackage{adjustbox}
\usepackage[pagestyles]{titlesec}
\usepackage{fancyhdr}
\usepackage{lipsum}
\usepackage[dvipsnames]{xcolor}
\usepackage{makecell}

% Đề tựa
\title{\Huge{\textbf{Bảng hợp đồng nhóm}}}
\author{Nhóm 1}
\author{Môn: Kĩ năng nghề nghiệp}
\date{\today}

% Xóa trang trắng
\begin{document}

\maketitle

\textcolor{red}{\chapter*{\Huge{{Nội dung}}}}

\noindent
{\color{cyan} \rule{\linewidth}{0.5mm} }

%\color{azure(colorwheel)}

\textcolor{red}{\section{\Large{Thành viên:}}}
%\section{{\Large{Thành viên:}}}

\begin{center}
    \begin{tabular}{|l|l|l|}
         \hline
         \color{red}{STT} & \color{red}{MSSV} & \color{red}{Họ và tên} \\
         \hline
         01 & 19520711 & Nguyễn Duy Mẫn \\
         \hline
         02 & 19520440 & Võ Nhật Cường \\
         \hline
         03 & 19520627 & Huỳnh Bảo Khánh \\
         \hline
         04 & 19520028 & Nguyễn Minh Cường \\
         \hline
         05 & 19520882 & Phạm Thanh Quang \\
         \hline
    \end{tabular}
\end{center}

\begin{itemize}
    \item \textbf{Nhóm trưởng: } Phạm Thanh Quang
\end{itemize}




\textcolor{red}{\section{\Large{Phân công công việc:}}}

\begin{enumerate}
    \item Mồi - hiển thị mồi: Nguyễn Duy Mẫn.
    \item Rắn: \par
        \begin{itemize}
            \item Khởi tạo + hiển thị + di chuyển: Võ Nhật Cường
            \item Ăn mồi + kiểm tra chết + kết thúc game: Huỳnh Bảo Khánh. 
        \end{itemize}
    \item Tường - vẽ - hiện thị: Nguyễn Minh Cường. 
    \item Khởi tạo game: Phạm Thanh Quang. 
\end{enumerate}




\textcolor{red}{\section{\Large{Mong muốn, kì vọng:}}}

\begin{table}[h]

\centering \def\arraystretch{1.5} \small

\begin{tabular}{|p{3cm}|p{10cm}|}

\hline

\textbf{Thành viên} & \textbf{ Mong muốn, kì vọng}\par \\ \hline

Nguyễn Duy Mẫn & Sẽ nâng cao được kĩ năng code, tư duy logic về vòng lặp trong game. Hoàn thiện hơn kĩ năng làm việc nhóm, giao tiếp, trao đổi, tương tác với mọi người.  \par\\ \hline

Võ Nhật Cường & Trau dồi thêm kĩ năng cần thiết trong làm việc nhóm, cách trình bày ý kiến, lắng nghe mọi người. Tạo động lực cho các thành viên và bản thân thêm đam mê với ngành IT.  \par\\ \hline

Huỳnh Bảo Khánh & Có thêm khả năng optimize game, sáng tạo thêm các tính năng mới như đánh boss, challenge round, duo rank. \par \\ \hline

Nguyễn Minh Cường & Rèn luyện kĩ năng làm việc nhóm, nâng cao khả năng quản lí thời gian. Cải thiện thêm kĩ năng đọc, hiểu code của người khác, và rèn luyện cách viết code dễ hiểu của bản thân. \par \\ \hline
    
Phạm Thanh Quang & Cải thiện khả năng trao đổi ý kiến với các thành viên, lắng nghe và tạo động lực, khuyến khích cả nhóm để đạt được một kết quả tốt nhất. Cả về nâng cao hiểu biết về điểm mạnh, yếu của người khác, hiểu người khác rõ hơn về tính cách của những người xung quanh mình. \par \\ \hline


\end{tabular}

\end{table}


\textcolor{red}{\section{\Large{Nội quy:}}}
\begin{enumerate}
    \large{
    \item Làm việc theo sự phân công của nhóm trưởng.
    \item Mỗi tuần phải báo cáo tiến độ làm việc với cả nhóm ít nhất một lần.
    \item Phải thêm description khi push code lên Github.
    \item Nếu có tạo branch mới để test thì phải thông báo với nhóm (ít nhất là với nhóm trưởng). 
    \item Để thêm tính năng mới vào game, phải bàn bạc trước với nhóm hoặc làm trước ở branch test. Nếu được nhóm đồng ý thì mới đẩy lên branch chính.
    \item Mỗi các nhân đều được tôn trọng với ý kiến và quan điểm của bản thân.
    \item Làm việc trên tinh thần luôn sẵn sàng giúp đỡ người khác và chủ động tìm kiếm sự giúp đỡ khi gặp khớ khăn.
    }
\end{enumerate}

\textcolor{red}{\section{\Large{Mục tiêu:}}}
\begin{enumerate}
    \large{
    \item Hoàn thành đồ án cùng với những kì vọng đã đề ra một cách tốt đẹp. Mỗi người đều đạt những kĩ năng mà bản thân mong muốn.
    \item Mỗi người có thể nhận ra thêm những điểm mạnh, yếu của khổng chỉ bản thân mà còn của người khác.
    \item Biết được thêm các kĩ năng giao tiếp, trao đổi ý kiến, thiết kế ý tưởng với cả nhóm. Biết cách chủ động giúp đỡ và nhờ người khác giúp mình không chỉ trong đồ án này, mà còn trong các công việc cho tương lai sau này.  
    \item Sáng tạo thêm các tính năng mới cho game.
    \item Làm quen với Github, phục vụ cho các đồ án sau này.

    }
    
\end{enumerate}

\textcolor{red}{\section{\Large{Tiêu chí đánh giá: }}}
\large{Dựa trên các tiêu chí cơ bản sau: } \\
\begin{enumerate}   
    \large{
    \item Hoàn thành các tasks được giao đúng tiến độ.
    \item Đóng góp tích cực cho nhóm: thường xuyên thảo luận, đưa ra đóng góp, ý kiến cho nhóm,...
    \item Thường xuyên báo cáo tiến độ làm việc, nhiều lần trong một tuần.
    \item Sẵn sàng hỗ trợ, giúp đỡ các thành viên khác.
    \item Giữ thái độ tích cực, giữ hòa khí trong nhóm.
    }
    
\end{enumerate}

\noindent
{\color{cyan} \rule{\linewidth}{0.5mm} }

\section*{\Large{Kí tên}}

\begin{figure}[h]

    \centering
    \includegraphics[width=0.3\paperwidth]{Man.png} \\
    \caption{Nguyễn Duy Mẫn}
    \label{fig:Christ1}
\end{figure}

\begin{figure}[h]

    \centering
    \includegraphics[width=0.3\paperwidth]{N.Cuong.png} \\
    \caption{Võ Nhật Cường}
    \label{fig:Christ1}
\end{figure}

\begin{figure}[h]

    \centering
    \includegraphics[width=0.3\paperwidth]{BKhanh.png} \\
    \caption{Huỳnh Bảo Khánh}
    \label{fig:Christ1}
\end{figure}


\begin{figure}[h]
    \centering
    \includegraphics[width=0.3\paperwidth]{M.Cuong.png} \\
    \caption{Nguyễn Minh Cường}
    \label{fig:Christ1}
\end{figure}

\begin{figure}[h]

    \centering
    \includegraphics[width=0.3\paperwidth]{Quang.png} \\
    \caption{Phạm Thanh Quang}
    \label{fig:Christ1}
\end{figure}




\end{document}
